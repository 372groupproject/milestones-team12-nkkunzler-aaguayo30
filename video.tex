\documentclass{article}
\usepackage[margin=1in,footskip=0.25in]{geometry}

\usepackage{listings}
\usepackage{graphicx}
\usepackage{amsmath}

\usepackage[utf8]{inputenc} % allow utf-8 input
\usepackage[T1]{fontenc}    % use 8-bit T1 fonts
\usepackage{hyperref}       % hyperlinks
\usepackage{url}            % simple URL typesetting
\usepackage{booktabs}       % professional-quality tables
\usepackage{amsfonts}       % blackboard math symbols
\usepackage{nicefrac}       % compact symbols for 1/2, etc.
\usepackage{microtype}      % microtypography
\usepackage{lipsum}

\begin{document}
\centering{
\large\textbf{x86 64-bit Program Video Ideas}
} 

\centering{
Nicholas Kunzler \\
Angel Aguayo
}

\begin{itemize}
    \item We will most likely be using a PowerPoint presentation that contains some of the main unique aspects of x86. In addition, the PowerPoint will contain short snippets of code and possible screen shots of the game in which we will introduce the libraries and code used to create the project. After the presentation, we will have a short game demo in which we show the functionality of our game.
    \item The main ideas that we would try to express is the limited nature of x86. This includes the limited number of control structures that exists, the minimal number of data types, and the way the language can be broken into sub programs.
    \item In order to give an overview of our code, we will most likely be showing the three major parts of the program and quickly discuss there purpose. These overviews will most likely orientate around the use of the Ncurses library to render ASCII characters to the terminal screen and how the game is loaded and interacted with.
    \item After a quick run through of the code base, we will screen record a demo of the game that shows two different possible game ending results. These results with either be a game won or game lose. Showing these should demonstrate the completeness of the game.
    \item We will most likely be using Zoom or some sort of video and audio capturing software in order to record the PowerPoint presentation and a voice over.
    \item Although this is our basic approach we may substitute ideas with new ones, so this may change in the next few days to weeks.
\end{itemize}

\end{document}
